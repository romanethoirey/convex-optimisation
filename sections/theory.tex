\documentclass[../convex_optimization.tex]{subfiles}
\begin{document}
This chapter will serve to introduce the concept of \ldots

TODO expand

\subsection{Problem formulation}
A convex optimization problem consists of, given a convex function
$f: S \to \mathbb R$ for a convex set $S$, finding
a point $x_0 \in S$ which minimizes the value $f(x_0) \in \mathbb R$.

Recall the definitions for convex sets and functions:
A set $S \subset \mathbb R^n$ is a \textbf{convex set} iff
\begin{equation}
    \forall (x, y) \in S^2: (1-\lambda)x + \lambda y \in S,
    \forall \lambda \in [0, 1]
    \label{convex_set}
\end{equation}
interpreted as any point on the line segment between two points in the set,
is also in the set.
A function $f$ is a \textbf{convex function on $S$} iff
\begin{equation}
    \forall (x, y) \in S^2, \forall \lambda \in [0, 1]:
    f((1-\lambda)x + \lambda y) \leq (1-\lambda)f(x) + \lambda f(y)
    \label{convex_function}
\end{equation}
thus the function value of a point between two contained points must
not exceed the value over the same point in the linear combination between the two contained points.

We will also recall that for a function of one variable,
twice differentiable on an interval $I$,
the function being \textbf{convex} is equivalent to its second derivative
being positive:
TODO: Check if this is mathematically sound
\begin{equation} \label{convex_corollary}
\begin{gathered}
    \text{For } f: \mathbb R \to \mathbb R
    \text{ with }\hat{f}: I \to \mathbb R \in C^2, \\
    f \textbf{ convex} \iff f''(x) \geq 0, x \in I
\end{gathered}
\end{equation}
When solving classical convex first-order or zero-order convex
optimization problems, these definition alone provide the solution.
For problems of second order or higher, these properties alone are
no longer sufficient.
\newpage
taken from the Wikipedia article~\cite{hessian_wiki}:

\begin{equation}
    \mathbf H = \begin{bmatrix}
        \dfrac{\partial^2 f}{\partial x_1^2} &
        \dfrac{\partial^2 f}{\partial x_1\,\partial x_2} &
        \cdots & \dfrac{\partial^2 f}{\partial x_1\,\partial x_n} \\[2.2ex]
        \dfrac{\partial^2 f}{\partial x_2\,\partial x_1} &
        \dfrac{\partial^2 f}{\partial x_2^2} & \cdots &
        \dfrac{\partial^2 f}{\partial x_2\,\partial x_n} \\[2.2ex]
        \vdots & \vdots & \ddots & \vdots \\[2.2ex]
        \dfrac{\partial^2 f}{\partial x_n\,\partial x_1} &
        \dfrac{\partial^2 f}{\partial x_n\,\partial x_2} &
        \cdots & \dfrac{\partial^2 f}{\partial x_n^2}
    \end{bmatrix}
    \label{hessian_definition}
\end{equation}
\end{document}
