\documentclass[../convex_optimization.tex]{subfiles}
    % Colors for the hyperref package
\definecolor{urlcolor}{rgb}{0,.145,.698}
\definecolor{linkcolor}{rgb}{.71,0.21,0.01}
\definecolor{citecolor}{rgb}{.12,.54,.11}

    % ANSI colors
\definecolor{ansi-black}{HTML}{3E424D}
\definecolor{ansi-black-intense}{HTML}{282C36}
\definecolor{ansi-red}{HTML}{E75C58}
\definecolor{ansi-red-intense}{HTML}{B22B31}
\definecolor{ansi-green}{HTML}{00A250}
\definecolor{ansi-green-intense}{HTML}{007427}
\definecolor{ansi-yellow}{HTML}{DDB62B}
\definecolor{ansi-yellow-intense}{HTML}{B27D12}
\definecolor{ansi-blue}{HTML}{208FFB}
\definecolor{ansi-blue-intense}{HTML}{0065CA}
\definecolor{ansi-magenta}{HTML}{D160C4}
\definecolor{ansi-magenta-intense}{HTML}{A03196}
\definecolor{ansi-cyan}{HTML}{60C6C8}
\definecolor{ansi-cyan-intense}{HTML}{258F8F}
\definecolor{ansi-white}{HTML}{C5C1B4}
\definecolor{ansi-white-intense}{HTML}{A1A6B2}
\definecolor{ansi-default-inverse-fg}{HTML}{FFFFFF}
\definecolor{ansi-default-inverse-bg}{HTML}{000000}

    % commands and environments needed by pandoc snippets
    % extracted from the output of `pandoc -s`
\providecommand{\tightlist}{%
\setlength{\itemsep}{0pt}\setlength{\parskip}{0pt}}
\DefineVerbatimEnvironment{Highlighting}{Verbatim}{commandchars=\\\{\}}
    % Add ',fontsize=\small' for more characters per line
\newenvironment{Shaded}{}{}
\newcommand{\KeywordTok}[1]{\textcolor[rgb]{0.00,0.44,0.13}{\textbf{{#1}}}}
\newcommand{\DataTypeTok}[1]{\textcolor[rgb]{0.56,0.13,0.00}{{#1}}}
\newcommand{\DecValTok}[1]{\textcolor[rgb]{0.25,0.63,0.44}{{#1}}}
\newcommand{\BaseNTok}[1]{\textcolor[rgb]{0.25,0.63,0.44}{{#1}}}
\newcommand{\FloatTok}[1]{\textcolor[rgb]{0.25,0.63,0.44}{{#1}}}
\newcommand{\CharTok}[1]{\textcolor[rgb]{0.25,0.44,0.63}{{#1}}}
\newcommand{\StringTok}[1]{\textcolor[rgb]{0.25,0.44,0.63}{{#1}}}
\newcommand{\CommentTok}[1]{\textcolor[rgb]{0.38,0.63,0.69}{\textit{{#1}}}}
\newcommand{\OtherTok}[1]{\textcolor[rgb]{0.00,0.44,0.13}{{#1}}}
\newcommand{\AlertTok}[1]{\textcolor[rgb]{1.00,0.00,0.00}{\textbf{{#1}}}}
\newcommand{\FunctionTok}[1]{\textcolor[rgb]{0.02,0.16,0.49}{{#1}}}
\newcommand{\RegionMarkerTok}[1]{{#1}}
\newcommand{\ErrorTok}[1]{\textcolor[rgb]{1.00,0.00,0.00}{\textbf{{#1}}}}
\newcommand{\NormalTok}[1]{{#1}}

    % Additional commands for more recent versions of Pandoc
\newcommand{\ConstantTok}[1]{\textcolor[rgb]{0.53,0.00,0.00}{{#1}}}
\newcommand{\SpecialCharTok}[1]{\textcolor[rgb]{0.25,0.44,0.63}{{#1}}}
\newcommand{\VerbatimStringTok}[1]{\textcolor[rgb]{0.25,0.44,0.63}{{#1}}}
\newcommand{\SpecialStringTok}[1]{\textcolor[rgb]{0.73,0.40,0.53}{{#1}}}
\newcommand{\ImportTok}[1]{{#1}}
\newcommand{\DocumentationTok}[1]{\textcolor[rgb]{0.73,0.13,0.13}{\textit{{#1}}}}
\newcommand{\AnnotationTok}[1]{\textcolor[rgb]{0.38,0.63,0.69}{\textbf{\textit{{#1}}}}}
\newcommand{\CommentVarTok}[1]{\textcolor[rgb]{0.38,0.63,0.69}{\textbf{\textit{{#1}}}}}
\newcommand{\VariableTok}[1]{\textcolor[rgb]{0.10,0.09,0.49}{{#1}}}
\newcommand{\ControlFlowTok}[1]{\textcolor[rgb]{0.00,0.44,0.13}{\textbf{{#1}}}}
\newcommand{\OperatorTok}[1]{\textcolor[rgb]{0.40,0.40,0.40}{{#1}}}
\newcommand{\BuiltInTok}[1]{{#1}}
\newcommand{\ExtensionTok}[1]{{#1}}
\newcommand{\PreprocessorTok}[1]{\textcolor[rgb]{0.74,0.48,0.00}{{#1}}}
\newcommand{\AttributeTok}[1]{\textcolor[rgb]{0.49,0.56,0.16}{{#1}}}
\newcommand{\InformationTok}[1]{\textcolor[rgb]{0.38,0.63,0.69}{\textbf{\textit{{#1}}}}}
\newcommand{\WarningTok}[1]{\textcolor[rgb]{0.38,0.63,0.69}{\textbf{\textit{{#1}}}}}

    % Define a nice break command that doesn't care if a line doesn't already
    % exist.
\def\br{\hspace*{\fill} \\* }
    % Math Jax compatibility definitions
\def\gt{>}
\def\lt{<}
\let\Oldtex\TeX
\let\Oldlatex\LaTeX
\renewcommand{\TeX}{\textrm{\Oldtex}}
\renewcommand{\LaTeX}{\textrm{\Oldlatex}}
    % Document parameters
    % Document title


% Pygments definitions
\makeatletter
\def\PY@reset{\let\PY@it=\relax \let\PY@bf=\relax%
    \let\PY@ul=\relax \let\PY@tc=\relax%
\let\PY@bc=\relax \let\PY@ff=\relax}
\def\PY@tok#1{\csname PY@tok@#1\endcsname}
\def\PY@toks#1+{\ifx\relax#1\empty\else%
\PY@tok{#1}\expandafter\PY@toks\fi}
\def\PY@do#1{\PY@bc{\PY@tc{\PY@ul{%
\PY@it{\PY@bf{\PY@ff{#1}}}}}}}
\def\PY#1#2{\PY@reset\PY@toks#1+\relax+\PY@do{#2}}

\expandafter\def\csname PY@tok@w\endcsname{\def\PY@tc##1{\textcolor[rgb]{0.73,0.73,0.73}{##1}}}
\expandafter\def\csname PY@tok@c\endcsname{\let\PY@it=\textit\def\PY@tc##1{\textcolor[rgb]{0.25,0.50,0.50}{##1}}}
\expandafter\def\csname PY@tok@cp\endcsname{\def\PY@tc##1{\textcolor[rgb]{0.74,0.48,0.00}{##1}}}
\expandafter\def\csname PY@tok@k\endcsname{\let\PY@bf=\textbf\def\PY@tc##1{\textcolor[rgb]{0.00,0.50,0.00}{##1}}}
\expandafter\def\csname PY@tok@kp\endcsname{\def\PY@tc##1{\textcolor[rgb]{0.00,0.50,0.00}{##1}}}
\expandafter\def\csname PY@tok@kt\endcsname{\def\PY@tc##1{\textcolor[rgb]{0.69,0.00,0.25}{##1}}}
\expandafter\def\csname PY@tok@o\endcsname{\def\PY@tc##1{\textcolor[rgb]{0.40,0.40,0.40}{##1}}}
\expandafter\def\csname PY@tok@ow\endcsname{\let\PY@bf=\textbf\def\PY@tc##1{\textcolor[rgb]{0.67,0.13,1.00}{##1}}}
\expandafter\def\csname PY@tok@nb\endcsname{\def\PY@tc##1{\textcolor[rgb]{0.00,0.50,0.00}{##1}}}
\expandafter\def\csname PY@tok@nf\endcsname{\def\PY@tc##1{\textcolor[rgb]{0.00,0.00,1.00}{##1}}}
\expandafter\def\csname PY@tok@nc\endcsname{\let\PY@bf=\textbf\def\PY@tc##1{\textcolor[rgb]{0.00,0.00,1.00}{##1}}}
\expandafter\def\csname PY@tok@nn\endcsname{\let\PY@bf=\textbf\def\PY@tc##1{\textcolor[rgb]{0.00,0.00,1.00}{##1}}}
\expandafter\def\csname PY@tok@ne\endcsname{\let\PY@bf=\textbf\def\PY@tc##1{\textcolor[rgb]{0.82,0.25,0.23}{##1}}}
\expandafter\def\csname PY@tok@nv\endcsname{\def\PY@tc##1{\textcolor[rgb]{0.10,0.09,0.49}{##1}}}
\expandafter\def\csname PY@tok@no\endcsname{\def\PY@tc##1{\textcolor[rgb]{0.53,0.00,0.00}{##1}}}
\expandafter\def\csname PY@tok@nl\endcsname{\def\PY@tc##1{\textcolor[rgb]{0.63,0.63,0.00}{##1}}}
\expandafter\def\csname PY@tok@ni\endcsname{\let\PY@bf=\textbf\def\PY@tc##1{\textcolor[rgb]{0.60,0.60,0.60}{##1}}}
\expandafter\def\csname PY@tok@na\endcsname{\def\PY@tc##1{\textcolor[rgb]{0.49,0.56,0.16}{##1}}}
\expandafter\def\csname PY@tok@nt\endcsname{\let\PY@bf=\textbf\def\PY@tc##1{\textcolor[rgb]{0.00,0.50,0.00}{##1}}}
\expandafter\def\csname PY@tok@nd\endcsname{\def\PY@tc##1{\textcolor[rgb]{0.67,0.13,1.00}{##1}}}
\expandafter\def\csname PY@tok@s\endcsname{\def\PY@tc##1{\textcolor[rgb]{0.73,0.13,0.13}{##1}}}
\expandafter\def\csname PY@tok@sd\endcsname{\let\PY@it=\textit\def\PY@tc##1{\textcolor[rgb]{0.73,0.13,0.13}{##1}}}
\expandafter\def\csname PY@tok@si\endcsname{\let\PY@bf=\textbf\def\PY@tc##1{\textcolor[rgb]{0.73,0.40,0.53}{##1}}}
\expandafter\def\csname PY@tok@se\endcsname{\let\PY@bf=\textbf\def\PY@tc##1{\textcolor[rgb]{0.73,0.40,0.13}{##1}}}
\expandafter\def\csname PY@tok@sr\endcsname{\def\PY@tc##1{\textcolor[rgb]{0.73,0.40,0.53}{##1}}}
\expandafter\def\csname PY@tok@ss\endcsname{\def\PY@tc##1{\textcolor[rgb]{0.10,0.09,0.49}{##1}}}
\expandafter\def\csname PY@tok@sx\endcsname{\def\PY@tc##1{\textcolor[rgb]{0.00,0.50,0.00}{##1}}}
\expandafter\def\csname PY@tok@m\endcsname{\def\PY@tc##1{\textcolor[rgb]{0.40,0.40,0.40}{##1}}}
\expandafter\def\csname PY@tok@gh\endcsname{\let\PY@bf=\textbf\def\PY@tc##1{\textcolor[rgb]{0.00,0.00,0.50}{##1}}}
\expandafter\def\csname PY@tok@gu\endcsname{\let\PY@bf=\textbf\def\PY@tc##1{\textcolor[rgb]{0.50,0.00,0.50}{##1}}}
\expandafter\def\csname PY@tok@gd\endcsname{\def\PY@tc##1{\textcolor[rgb]{0.63,0.00,0.00}{##1}}}
\expandafter\def\csname PY@tok@gi\endcsname{\def\PY@tc##1{\textcolor[rgb]{0.00,0.63,0.00}{##1}}}
\expandafter\def\csname PY@tok@gr\endcsname{\def\PY@tc##1{\textcolor[rgb]{1.00,0.00,0.00}{##1}}}
\expandafter\def\csname PY@tok@ge\endcsname{\let\PY@it=\textit}
\expandafter\def\csname PY@tok@gs\endcsname{\let\PY@bf=\textbf}
\expandafter\def\csname PY@tok@gp\endcsname{\let\PY@bf=\textbf\def\PY@tc##1{\textcolor[rgb]{0.00,0.00,0.50}{##1}}}
\expandafter\def\csname PY@tok@go\endcsname{\def\PY@tc##1{\textcolor[rgb]{0.53,0.53,0.53}{##1}}}
\expandafter\def\csname PY@tok@gt\endcsname{\def\PY@tc##1{\textcolor[rgb]{0.00,0.27,0.87}{##1}}}
\expandafter\def\csname PY@tok@err\endcsname{\def\PY@bc##1{\setlength{\fboxsep}{0pt}\fcolorbox[rgb]{1.00,0.00,0.00}{1,1,1}{\strut ##1}}}
\expandafter\def\csname PY@tok@kc\endcsname{\let\PY@bf=\textbf\def\PY@tc##1{\textcolor[rgb]{0.00,0.50,0.00}{##1}}}
\expandafter\def\csname PY@tok@kd\endcsname{\let\PY@bf=\textbf\def\PY@tc##1{\textcolor[rgb]{0.00,0.50,0.00}{##1}}}
\expandafter\def\csname PY@tok@kn\endcsname{\let\PY@bf=\textbf\def\PY@tc##1{\textcolor[rgb]{0.00,0.50,0.00}{##1}}}
\expandafter\def\csname PY@tok@kr\endcsname{\let\PY@bf=\textbf\def\PY@tc##1{\textcolor[rgb]{0.00,0.50,0.00}{##1}}}
\expandafter\def\csname PY@tok@bp\endcsname{\def\PY@tc##1{\textcolor[rgb]{0.00,0.50,0.00}{##1}}}
\expandafter\def\csname PY@tok@fm\endcsname{\def\PY@tc##1{\textcolor[rgb]{0.00,0.00,1.00}{##1}}}
\expandafter\def\csname PY@tok@vc\endcsname{\def\PY@tc##1{\textcolor[rgb]{0.10,0.09,0.49}{##1}}}
\expandafter\def\csname PY@tok@vg\endcsname{\def\PY@tc##1{\textcolor[rgb]{0.10,0.09,0.49}{##1}}}
\expandafter\def\csname PY@tok@vi\endcsname{\def\PY@tc##1{\textcolor[rgb]{0.10,0.09,0.49}{##1}}}
\expandafter\def\csname PY@tok@vm\endcsname{\def\PY@tc##1{\textcolor[rgb]{0.10,0.09,0.49}{##1}}}
\expandafter\def\csname PY@tok@sa\endcsname{\def\PY@tc##1{\textcolor[rgb]{0.73,0.13,0.13}{##1}}}
\expandafter\def\csname PY@tok@sb\endcsname{\def\PY@tc##1{\textcolor[rgb]{0.73,0.13,0.13}{##1}}}
\expandafter\def\csname PY@tok@sc\endcsname{\def\PY@tc##1{\textcolor[rgb]{0.73,0.13,0.13}{##1}}}
\expandafter\def\csname PY@tok@dl\endcsname{\def\PY@tc##1{\textcolor[rgb]{0.73,0.13,0.13}{##1}}}
\expandafter\def\csname PY@tok@s2\endcsname{\def\PY@tc##1{\textcolor[rgb]{0.73,0.13,0.13}{##1}}}
\expandafter\def\csname PY@tok@sh\endcsname{\def\PY@tc##1{\textcolor[rgb]{0.73,0.13,0.13}{##1}}}
\expandafter\def\csname PY@tok@s1\endcsname{\def\PY@tc##1{\textcolor[rgb]{0.73,0.13,0.13}{##1}}}
\expandafter\def\csname PY@tok@mb\endcsname{\def\PY@tc##1{\textcolor[rgb]{0.40,0.40,0.40}{##1}}}
\expandafter\def\csname PY@tok@mf\endcsname{\def\PY@tc##1{\textcolor[rgb]{0.40,0.40,0.40}{##1}}}
\expandafter\def\csname PY@tok@mh\endcsname{\def\PY@tc##1{\textcolor[rgb]{0.40,0.40,0.40}{##1}}}
\expandafter\def\csname PY@tok@mi\endcsname{\def\PY@tc##1{\textcolor[rgb]{0.40,0.40,0.40}{##1}}}
\expandafter\def\csname PY@tok@il\endcsname{\def\PY@tc##1{\textcolor[rgb]{0.40,0.40,0.40}{##1}}}
\expandafter\def\csname PY@tok@mo\endcsname{\def\PY@tc##1{\textcolor[rgb]{0.40,0.40,0.40}{##1}}}
\expandafter\def\csname PY@tok@ch\endcsname{\let\PY@it=\textit\def\PY@tc##1{\textcolor[rgb]{0.25,0.50,0.50}{##1}}}
\expandafter\def\csname PY@tok@cm\endcsname{\let\PY@it=\textit\def\PY@tc##1{\textcolor[rgb]{0.25,0.50,0.50}{##1}}}
\expandafter\def\csname PY@tok@cpf\endcsname{\let\PY@it=\textit\def\PY@tc##1{\textcolor[rgb]{0.25,0.50,0.50}{##1}}}
\expandafter\def\csname PY@tok@c1\endcsname{\let\PY@it=\textit\def\PY@tc##1{\textcolor[rgb]{0.25,0.50,0.50}{##1}}}
\expandafter\def\csname PY@tok@cs\endcsname{\let\PY@it=\textit\def\PY@tc##1{\textcolor[rgb]{0.25,0.50,0.50}{##1}}}

\def\PYZbs{\char`\\}
\def\PYZus{\char`\_}
\def\PYZob{\char`\{}
\def\PYZcb{\char`\}}
\def\PYZca{\char`\^}
\def\PYZam{\char`\&}
\def\PYZlt{\char`\<}
\def\PYZgt{\char`\>}
\def\PYZsh{\char`\#}
\def\PYZpc{\char`\%}
\def\PYZdl{\char`\$}
\def\PYZhy{\char`\-}
\def\PYZsq{\char`\'}
\def\PYZdq{\char`\"}
\def\PYZti{\char`\~}
% for compatibility with earlier versions
\def\PYZat{@}
\def\PYZlb{[}
\def\PYZrb{]}
\makeatother


    % For linebreaks inside Verbatim environment from package fancyvrb. 
\makeatletter
\newbox\Wrappedcontinuationbox 
\newbox\Wrappedvisiblespacebox 
\newcommand*\Wrappedvisiblespace {\textcolor{red}{\textvisiblespace}} 
\newcommand*\Wrappedcontinuationsymbol {\textcolor{red}{\llap{\tiny$\m@th\hookrightarrow$}}} 
\newcommand*\Wrappedcontinuationindent {3ex } 
\newcommand*\Wrappedafterbreak {\kern\Wrappedcontinuationindent\copy\Wrappedcontinuationbox} 
        % Take advantage of the already applied Pygments mark-up to insert 
        % potential linebreaks for TeX processing. 
        %        {, <, #, %, $, ' and ": go to next line. 
        %        _, }, ^, &, >, - and ~: stay at end of broken line. 
        % Use of \textquotesingle for straight quote. 
\newcommand*\Wrappedbreaksatspecials {% 
    \def\PYGZus{\discretionary{\char`\_}{\Wrappedafterbreak}{\char`\_}}% 
    \def\PYGZob{\discretionary{}{\Wrappedafterbreak\char`\{}{\char`\{}}% 
    \def\PYGZcb{\discretionary{\char`\}}{\Wrappedafterbreak}{\char`\}}}% 
    \def\PYGZca{\discretionary{\char`\^}{\Wrappedafterbreak}{\char`\^}}% 
    \def\PYGZam{\discretionary{\char`\&}{\Wrappedafterbreak}{\char`\&}}% 
    \def\PYGZlt{\discretionary{}{\Wrappedafterbreak\char`\<}{\char`\<}}% 
    \def\PYGZgt{\discretionary{\char`\>}{\Wrappedafterbreak}{\char`\>}}% 
    \def\PYGZsh{\discretionary{}{\Wrappedafterbreak\char`\#}{\char`\#}}% 
    \def\PYGZpc{\discretionary{}{\Wrappedafterbreak\char`\%}{\char`\%}}% 
    \def\PYGZdl{\discretionary{}{\Wrappedafterbreak\char`\$}{\char`\$}}% 
    \def\PYGZhy{\discretionary{\char`\-}{\Wrappedafterbreak}{\char`\-}}% 
    \def\PYGZsq{\discretionary{}{\Wrappedafterbreak\textquotesingle}{\textquotesingle}}% 
    \def\PYGZdq{\discretionary{}{\Wrappedafterbreak\char`\"}{\char`\"}}% 
    \def\PYGZti{\discretionary{\char`\~}{\Wrappedafterbreak}{\char`\~}}% 
} 
        % Some characters . , ; ? ! / are not pygmentized. 
        % This macro makes them "active" and they will insert potential linebreaks 
\newcommand*\Wrappedbreaksatpunct {% 
    \lccode`\~`\.\lowercase{\def~}{\discretionary{\hbox{\char`\.}}{\Wrappedafterbreak}{\hbox{\char`\.}}}% 
    \lccode`\~`\,\lowercase{\def~}{\discretionary{\hbox{\char`\,}}{\Wrappedafterbreak}{\hbox{\char`\,}}}% 
    \lccode`\~`\;\lowercase{\def~}{\discretionary{\hbox{\char`\;}}{\Wrappedafterbreak}{\hbox{\char`\;}}}% 
    \lccode`\~`\:\lowercase{\def~}{\discretionary{\hbox{\char`\:}}{\Wrappedafterbreak}{\hbox{\char`\:}}}% 
    \lccode`\~`\?\lowercase{\def~}{\discretionary{\hbox{\char`\?}}{\Wrappedafterbreak}{\hbox{\char`\?}}}% 
    \lccode`\~`\!\lowercase{\def~}{\discretionary{\hbox{\char`\!}}{\Wrappedafterbreak}{\hbox{\char`\!}}}% 
    \lccode`\~`\/\lowercase{\def~}{\discretionary{\hbox{\char`\/}}{\Wrappedafterbreak}{\hbox{\char`\/}}}% 
    \catcode`\.\active
    \catcode`\,\active 
    \catcode`\;\active
    \catcode`\:\active
    \catcode`\?\active
    \catcode`\!\active
    \catcode`\/\active 
    \lccode`\~`\~ 	
}
\makeatother

\let\OriginalVerbatim=\Verbatim
\makeatletter
\renewcommand{\Verbatim}[1][1]{%
        %\parskip\z@skip
    \sbox\Wrappedcontinuationbox {\Wrappedcontinuationsymbol}%
    \sbox\Wrappedvisiblespacebox {\FV@SetupFont\Wrappedvisiblespace}%
    \def\FancyVerbFormatLine ##1{\hsize\linewidth
        \vtop{\raggedright\hyphenpenalty\z@\exhyphenpenalty\z@
            \doublehyphendemerits\z@\finalhyphendemerits\z@
        \strut ##1\strut}%
    }%
        % If the linebreak is at a space, the latter will be displayed as visible
        % space at end of first line, and a continuation symbol starts next line.
        % Stretch/shrink are however usually zero for typewriter font.
    \def\FV@Space {%
        \nobreak\hskip\z@ plus\fontdimen3\font minus\fontdimen4\font
        \discretionary{\copy\Wrappedvisiblespacebox}{\Wrappedafterbreak}
        {\kern\fontdimen2\font}%
    }%

        % Allow breaks at special characters using \PYG... macros.
    \Wrappedbreaksatspecials
        % Breaks at punctuation characters . , ; ? ! and / need catcode=\active 	
    \OriginalVerbatim[#1,codes*=\Wrappedbreaksatpunct]%
}
\makeatother

    % Exact colors from NB
\definecolor{incolor}{HTML}{303F9F}
\definecolor{outcolor}{HTML}{D84315}
\definecolor{cellborder}{HTML}{CFCFCF}
\definecolor{cellbackground}{HTML}{F7F7F7}

    % prompt
\newcommand{\prompt}[4]{
    \llap{{\color{#2}[#3]: #4}}\vspace{-1.25em}
}



    % Prevent overflowing lines due to hard-to-break entities
\sloppy 
    % Setup hyperref package
\hypersetup{
    breaklinks=true,  % so long urls are correctly broken across lines
    colorlinks=true,
    urlcolor=urlcolor,
    linkcolor=linkcolor,
    citecolor=citecolor,
}
    % Slightly bigger margins than the latex defaults

\begin{document}

We are going to implement a solver for optimisation problems using
Hessian matrices. We will use the two functions that we have also
studied analytically.

\hypertarget{importation-of-libraries}{%
\subsection{Importation of libraries}\label{importation-of-libraries}}

We use the \emph{SymPy} library to compute and display our function.

\begin{tcolorbox}[breakable, size=fbox, boxrule=1pt, pad at break*=1mm,colback=cellbackground, colframe=cellborder]
    \prompt{In}{incolor}{1}{\hspace{4pt}}
    \begin{Verbatim}[commandchars=\\\{\}]
        \PY{k+kn}{import} \PY{n+nn}{pandas} \PY{k}{as} \PY{n+nn}{pd}
        \PY{k+kn}{import} \PY{n+nn}{numpy} \PY{k}{as} \PY{n+nn}{np}
        \PY{k+kn}{import} \PY{n+nn}{re}
        \PY{k+kn}{from} \PY{n+nn}{sympy} \PY{k}{import} \PY{o}{*}
        \PY{k+kn}{import} \PY{n+nn}{math}
    \end{Verbatim}
\end{tcolorbox}

\begin{tcolorbox}[breakable, size=fbox, boxrule=1pt, pad at break*=1mm,colback=cellbackground, colframe=cellborder]
    \prompt{In}{incolor}{2}{\hspace{4pt}}
    \begin{Verbatim}[commandchars=\\\{\}]
        \PY{c+c1}{\PYZsh{} Showing how to extract symbols}
        \PY{n}{x}\PY{p}{,} \PY{n}{y}\PY{p}{,} \PY{n}{z} \PY{o}{=} \PY{n}{symbols}\PY{p}{(}\PY{l+s+s1}{\PYZsq{}}\PY{l+s+s1}{x, y, z}\PY{l+s+s1}{\PYZsq{}}\PY{p}{)}
        \PY{n}{multivar\PYZus{}exp} \PY{o}{=} \PY{n+nb}{eval}\PY{p}{(}\PY{l+s+s1}{\PYZsq{}}\PY{l+s+s1}{z + y + x}\PY{l+s+s1}{\PYZsq{}}\PY{p}{)}
        \PY{n+nb}{print}\PY{p}{(}\PY{n}{multivar\PYZus{}exp}\PY{o}{.}\PY{n}{free\PYZus{}symbols}\PY{p}{)}

        \PY{n}{one}\PY{p}{,} \PY{n}{two}\PY{p}{,} \PY{n}{three} \PY{o}{=} \PY{n}{symbols}\PY{p}{(}\PY{l+s+s1}{\PYZsq{}}\PY{l+s+s1}{x\PYZus{}1 x\PYZus{}2 x\PYZus{}3}\PY{l+s+s1}{\PYZsq{}}\PY{p}{)}
        \PY{n}{multivar\PYZus{}exp} \PY{o}{=} \PY{n+nb}{eval}\PY{p}{(}\PY{l+s+s1}{\PYZsq{}}\PY{l+s+s1}{three + two + one}\PY{l+s+s1}{\PYZsq{}}\PY{p}{)}
        \PY{n+nb}{print}\PY{p}{(}\PY{l+s+s1}{\PYZsq{}}\PY{l+s+s1}{Free symbols:}\PY{l+s+s1}{\PYZsq{}}\PY{p}{,} \PY{n}{multivar\PYZus{}exp}\PY{o}{.}\PY{n}{free\PYZus{}symbols}\PY{p}{)}

        \PY{c+c1}{\PYZsh{} Showing how indices are rendered}
        \PY{n+nb}{print}\PY{p}{(}\PY{l+s+s1}{\PYZsq{}}\PY{l+s+s1}{Rendering the expression:}\PY{l+s+s1}{\PYZsq{}}\PY{p}{,} \PY{n}{multivar\PYZus{}exp}\PY{p}{)}
    \end{Verbatim}
\end{tcolorbox}

\begin{Verbatim}[commandchars=\\\{\}]
    \{z, y, x\}
    Free symbols: \{x\_1, x\_3, x\_2\}
    Rendering the expression: x\_1 + x\_2 + x\_3
\end{Verbatim}

\begin{tcolorbox}[breakable, size=fbox, boxrule=1pt, pad at break*=1mm,colback=cellbackground, colframe=cellborder]
    \prompt{In}{incolor}{3}{\hspace{4pt}}
    \begin{Verbatim}[commandchars=\\\{\}]
        \PY{c+c1}{\PYZsh{} Fixing ugly printing}
        \PY{n}{init\PYZus{}printing}\PY{p}{(}\PY{p}{)}
        \PY{n}{display}\PY{p}{(}\PY{l+s+s1}{\PYZsq{}}\PY{l+s+s1}{Pretty rendering with display():}\PY{l+s+s1}{\PYZsq{}}\PY{p}{,} \PY{n}{multivar\PYZus{}exp}\PY{p}{)}
    \end{Verbatim}
\end{tcolorbox}


\begin{verbatim}
'Pretty rendering with display():'
\end{verbatim}


$\displaystyle x_{1} + x_{2} + x_{3}$


\hypertarget{implementation-of-functions}{%
    \subsection{Implementation of
functions}\label{implementation-of-functions}}

Here we implement few functions to calculate what we need to perform the
optimization problem with the Hessian Matrix.

\begin{tcolorbox}[breakable, size=fbox, boxrule=1pt, pad at break*=1mm,colback=cellbackground, colframe=cellborder]
    \prompt{In}{incolor}{4}{\hspace{4pt}}
    \begin{Verbatim}[commandchars=\\\{\}]
        \PY{k}{def} \PY{n+nf}{evaluate\PYZus{}value}\PY{p}{(}\PY{n}{func}\PY{p}{,} \PY{n}{symbol\PYZus{}list}\PY{p}{,} \PY{n}{point}\PY{p}{)}\PY{p}{:}
        \PY{l+s+sd}{\PYZdq{}\PYZdq{}\PYZdq{}}
        \PY{l+s+sd}{    Takes point tuple for evaluation, with variables in alphabetical order.}
        \PY{l+s+sd}{    fails if all symbols aren\PYZsq{}t mapped, returns numerical evaluation}
        \PY{l+s+sd}{    \PYZdq{}\PYZdq{}\PYZdq{}}
        \PY{k}{if} \PY{n+nb}{len}\PY{p}{(}\PY{n}{symbol\PYZus{}list}\PY{p}{)} \PY{o}{!=} \PY{n+nb}{len}\PY{p}{(}\PY{n}{point}\PY{p}{)}\PY{p}{:}
        \PY{n+nb}{print}\PY{p}{(}\PY{n}{symbol\PYZus{}list}\PY{p}{)}
        \PY{n+nb}{print}\PY{p}{(}\PY{n}{point}\PY{p}{)}
        \PY{k}{raise} \PY{n+ne}{ValueError}\PY{p}{(}\PY{l+s+s1}{\PYZsq{}}\PY{l+s+s1}{Unmatched argument sizes}\PY{l+s+s1}{\PYZsq{}}\PY{p}{)}
        \PY{n}{result} \PY{o}{=} \PY{n}{func}\PY{o}{.}\PY{n}{copy}\PY{p}{(}\PY{p}{)}
        \PY{k}{for} \PY{n}{symbol}\PY{p}{,} \PY{n}{value} \PY{o+ow}{in} \PY{n+nb}{zip}\PY{p}{(}\PY{n}{symbol\PYZus{}list}\PY{p}{,} \PY{n}{point}\PY{p}{)}\PY{p}{:}
        \PY{n}{result} \PY{o}{=} \PY{n}{result}\PY{o}{.}\PY{n}{subs}\PY{p}{(}\PY{n}{symbol}\PY{p}{,} \PY{n}{value}\PY{p}{)}
        \PY{k}{return} \PY{n}{result}

        \PY{n}{x} \PY{o}{=} \PY{n}{Symbol}\PY{p}{(}\PY{l+s+s1}{\PYZsq{}}\PY{l+s+s1}{x}\PY{l+s+s1}{\PYZsq{}}\PY{p}{)}
        \PY{n}{y} \PY{o}{=} \PY{n}{Symbol}\PY{p}{(}\PY{l+s+s1}{\PYZsq{}}\PY{l+s+s1}{y}\PY{l+s+s1}{\PYZsq{}}\PY{p}{)}
        \PY{n}{example\PYZus{}func} \PY{o}{=} \PY{n+nb}{eval}\PY{p}{(}\PY{l+s+s1}{\PYZsq{}}\PY{l+s+s1}{exp(x\PYZhy{}y)*(x**2\PYZhy{}2*y**2+4*x+2)}\PY{l+s+s1}{\PYZsq{}}\PY{p}{)}
        \PY{n}{syms} \PY{o}{=} \PY{n+nb}{list}\PY{p}{(}\PY{n}{example\PYZus{}func}\PY{o}{.}\PY{n}{free\PYZus{}symbols}\PY{p}{)}
        \PY{n}{display}\PY{p}{(}\PY{n}{example\PYZus{}func}\PY{p}{,} \PY{n}{syms}\PY{p}{,} \PY{l+s+s1}{\PYZsq{}}\PY{l+s+s1}{ with (\PYZhy{}4, \PYZhy{}2):}\PY{l+s+s1}{\PYZsq{}}\PY{p}{)}
        \PY{n}{display}\PY{p}{(}\PY{n}{evaluate\PYZus{}value}\PY{p}{(}\PY{n}{example\PYZus{}func}\PY{p}{,} \PY{n}{syms}\PY{p}{,} \PY{p}{(}\PY{o}{\PYZhy{}}\PY{l+m+mi}{4}\PY{p}{,} \PY{o}{\PYZhy{}}\PY{l+m+mi}{2}\PY{p}{)}\PY{p}{)}\PY{p}{)}

    \end{Verbatim}
\end{tcolorbox}

$\displaystyle \left(x^{2} + 4 x - 2 y^{2} + 2\right) e^{x - y}$


$\displaystyle \left[ y, \  x\right]$



\begin{verbatim}
' with (-4, -2):'
\end{verbatim}


$\displaystyle - 34 e^{2}$


\begin{tcolorbox}[breakable, size=fbox, boxrule=1pt, pad at break*=1mm,colback=cellbackground, colframe=cellborder]
    \prompt{In}{incolor}{5}{\hspace{4pt}}
    \begin{Verbatim}[commandchars=\\\{\}]
        \PY{k}{def} \PY{n+nf}{gradient}\PY{p}{(}\PY{n}{func}\PY{p}{,} \PY{n}{symbol\PYZus{}list}\PY{p}{)}\PY{p}{:}
        \PY{l+s+sd}{\PYZdq{}\PYZdq{}\PYZdq{} Returns list of derivatives for each free symbol (in alphabetical order) \PYZdq{}\PYZdq{}\PYZdq{}}
        \PY{n}{partial\PYZus{}derivatives} \PY{o}{=} \PY{p}{[}\PY{p}{]}
        \PY{k}{for} \PY{n}{symbol} \PY{o+ow}{in} \PY{n}{symbol\PYZus{}list}\PY{p}{:}
        \PY{n}{partial\PYZus{}derivatives}\PY{o}{.}\PY{n}{append}\PY{p}{(}\PY{n}{diff}\PY{p}{(}\PY{n}{func}\PY{p}{,} \PY{n}{symbol}\PY{p}{)}\PY{p}{)}
        \PY{k}{return} \PY{n}{partial\PYZus{}derivatives}

        \PY{n}{x1}\PY{p}{,} \PY{n}{x2} \PY{o}{=} \PY{n}{symbols}\PY{p}{(}\PY{l+s+s1}{\PYZsq{}}\PY{l+s+s1}{x y}\PY{l+s+s1}{\PYZsq{}}\PY{p}{)}
        \PY{n}{example\PYZus{}func} \PY{o}{=} \PY{n+nb}{eval}\PY{p}{(}\PY{l+s+s1}{\PYZsq{}}\PY{l+s+s1}{exp(x\PYZhy{}y)*(x**2\PYZhy{}2*y**2)}\PY{l+s+s1}{\PYZsq{}}\PY{p}{)}
        \PY{n+nb}{print}\PY{p}{(}\PY{l+s+s1}{\PYZsq{}}\PY{l+s+s1}{Gradient example using:}\PY{l+s+s1}{\PYZsq{}}\PY{p}{)}
        \PY{n}{display}\PY{p}{(}\PY{n}{example\PYZus{}func}\PY{p}{)}
        \PY{n+nb}{print}\PY{p}{(}\PY{l+s+s1}{\PYZsq{}}\PY{l+s+s1}{All partial derivatives:}\PY{l+s+s1}{\PYZsq{}}\PY{p}{)}
        \PY{n}{display}\PY{p}{(}\PY{n}{gradient}\PY{p}{(}\PY{n}{example\PYZus{}func}\PY{p}{,} \PY{n}{example\PYZus{}func}\PY{o}{.}\PY{n}{free\PYZus{}symbols}\PY{p}{)}\PY{p}{)}
    \end{Verbatim}
\end{tcolorbox}

\begin{Verbatim}[commandchars=\\\{\}]
    Gradient example using:
\end{Verbatim}

$\displaystyle \left(x^{2} - 2 y^{2}\right) e^{x - y}$


\begin{Verbatim}[commandchars=\\\{\}]
    All partial derivatives:
\end{Verbatim}

$\displaystyle \left[ - 4 y e^{x - y} - \left(x^{2} - 2 y^{2}\right) e^{x - y}, \  2 x e^{x - y} + \left(x^{2} - 2 y^{2}\right) e^{x - y}\right]$


\begin{tcolorbox}[breakable, size=fbox, boxrule=1pt, pad at break*=1mm,colback=cellbackground, colframe=cellborder]
    \prompt{In}{incolor}{6}{\hspace{4pt}}
    \begin{Verbatim}[commandchars=\\\{\}]
        \PY{k}{def} \PY{n+nf}{calcul\PYZus{}hessian\PYZus{}matrix}\PY{p}{(}\PY{n}{func}\PY{p}{)}\PY{p}{:}
        \PY{l+s+sd}{\PYZdq{}\PYZdq{}\PYZdq{}}
        \PY{l+s+sd}{    Calculate the hessian matrix of the function.}
        \PY{l+s+sd}{    \PYZdq{}\PYZdq{}\PYZdq{}}
        \PY{n}{syms} \PY{o}{=} \PY{n}{func}\PY{o}{.}\PY{n}{free\PYZus{}symbols}
        \PY{n}{partial\PYZus{}derivates} \PY{o}{=} \PY{n}{gradient}\PY{p}{(}\PY{n}{func}\PY{p}{,} \PY{n}{syms}\PY{p}{)}
        \PY{n}{partial\PYZus{}derivates\PYZus{}second} \PY{o}{=} \PY{p}{[}\PY{p}{]}

        \PY{k}{for} \PY{n}{derivate} \PY{o+ow}{in} \PY{n}{partial\PYZus{}derivates}\PY{p}{:}
        \PY{n}{partial\PYZus{}derivates\PYZus{}second}\PY{o}{.}\PY{n}{append}\PY{p}{(}\PY{n}{factor}\PY{p}{(}\PY{n}{gradient}\PY{p}{(}\PY{n}{derivate}\PY{p}{,} \PY{n}{syms}\PY{p}{)}\PY{p}{)}\PY{p}{)}
        \PY{n}{hessian\PYZus{}matrix} \PY{o}{=} \PY{n}{Matrix}\PY{p}{(}\PY{n}{partial\PYZus{}derivates\PYZus{}second}\PY{p}{)}
        \PY{k}{return} \PY{n}{hessian\PYZus{}matrix}
    \end{Verbatim}
\end{tcolorbox}

\begin{tcolorbox}[breakable, size=fbox, boxrule=1pt, pad at break*=1mm,colback=cellbackground, colframe=cellborder]
    \prompt{In}{incolor}{7}{\hspace{4pt}}
    \begin{Verbatim}[commandchars=\\\{\}]
        \PY{k}{def} \PY{n+nf}{calculate\PYZus{}determinant}\PY{p}{(}\PY{n}{func}\PY{p}{,} \PY{n}{point}\PY{p}{)}\PY{p}{:}
        \PY{l+s+sd}{\PYZdq{}\PYZdq{}\PYZdq{}}
        \PY{l+s+sd}{    Calculate the determinant based on the hessian matrix of the function.}
        \PY{l+s+sd}{    \PYZdq{}\PYZdq{}\PYZdq{}}
        \PY{n}{syms} \PY{o}{=} \PY{n}{func}\PY{o}{.}\PY{n}{free\PYZus{}symbols}
        \PY{k}{return} \PY{n}{evaluate\PYZus{}value}\PY{p}{(}\PY{n}{calcul\PYZus{}hessian\PYZus{}matrix}\PY{p}{(}\PY{n}{func}\PY{p}{)}\PY{p}{,} \PY{n}{syms}\PY{p}{,} \PY{n}{point}\PY{p}{)}\PY{o}{.}\PY{n}{det}\PY{p}{(}\PY{p}{)}
    \end{Verbatim}
\end{tcolorbox}

\begin{tcolorbox}[breakable, size=fbox, boxrule=1pt, pad at break*=1mm,colback=cellbackground, colframe=cellborder]
    \prompt{In}{incolor}{8}{\hspace{4pt}}
    \begin{Verbatim}[commandchars=\\\{\}]
        \PY{k}{def} \PY{n+nf}{calculate\PYZus{}trace}\PY{p}{(}\PY{n}{func}\PY{p}{,} \PY{n}{point}\PY{p}{)}\PY{p}{:}
        \PY{l+s+sd}{\PYZdq{}\PYZdq{}\PYZdq{}}
        \PY{l+s+sd}{    Calculate the trace based on the hessian matrix of the function.}
        \PY{l+s+sd}{    \PYZdq{}\PYZdq{}\PYZdq{}}
        \PY{n}{hessian} \PY{o}{=} \PY{n}{calcul\PYZus{}hessian\PYZus{}matrix}\PY{p}{(}\PY{n}{func}\PY{p}{)}
        \PY{n}{syms} \PY{o}{=} \PY{n}{func}\PY{o}{.}\PY{n}{free\PYZus{}symbols}
        \PY{n}{eigenvals} \PY{o}{=} \PY{n}{evaluate\PYZus{}value}\PY{p}{(}\PY{n}{hessian}\PY{p}{,} \PY{n}{syms}\PY{p}{,} \PY{n}{point}\PY{p}{)}\PY{o}{.}\PY{n}{eigenvals}\PY{p}{(}\PY{p}{)}
        \PY{c+c1}{\PYZsh{} Trace definition, multiplying for multiplicity}
        \PY{n}{trace} \PY{o}{=} \PY{n+nb}{sum}\PY{p}{(}\PY{p}{[}\PY{n}{k}\PY{o}{*}\PY{n}{v} \PY{k}{for} \PY{n}{k}\PY{p}{,} \PY{n}{v} \PY{o+ow}{in} \PY{n}{eigenvals}\PY{o}{.}\PY{n}{items}\PY{p}{(}\PY{p}{)}\PY{p}{]}\PY{p}{)}
        \PY{k}{return} \PY{n}{trace}\PY{o}{.}\PY{n}{simplify}\PY{p}{(}\PY{p}{)}
    \end{Verbatim}
\end{tcolorbox}

\begin{tcolorbox}[breakable, size=fbox, boxrule=1pt, pad at break*=1mm,colback=cellbackground, colframe=cellborder]
    \prompt{In}{incolor}{9}{\hspace{4pt}}
    \begin{Verbatim}[commandchars=\\\{\}]
        \PY{k}{def} \PY{n+nf}{is\PYZus{}critical\PYZus{}point}\PY{p}{(}\PY{n}{func}\PY{p}{,} \PY{n}{point}\PY{p}{)}\PY{p}{:}
        \PY{n}{hessian} \PY{o}{=} \PY{n}{calcul\PYZus{}hessian\PYZus{}matrix}\PY{p}{(}\PY{n}{func}\PY{p}{)}
        \PY{k}{if} \PY{n}{calculate\PYZus{}determinant}\PY{p}{(}\PY{n}{func}\PY{p}{,} \PY{n}{point}\PY{p}{)} \PY{o}{\PYZgt{}} \PY{l+m+mi}{0}\PY{p}{:}
        \PY{n}{trace} \PY{o}{=} \PY{n}{calculate\PYZus{}trace}\PY{p}{(}\PY{n}{func}\PY{p}{,} \PY{n}{point}\PY{p}{)}
        \PY{k}{if} \PY{n}{trace} \PY{o}{\PYZgt{}} \PY{l+m+mi}{0} \PY{p}{:}
        \PY{k}{return} \PY{l+s+s1}{\PYZsq{}}\PY{l+s+s1}{The function has a minimum.}\PY{l+s+s1}{\PYZsq{}}
        \PY{k}{elif} \PY{n}{trace} \PY{o}{\PYZlt{}} \PY{l+m+mi}{0}\PY{p}{:}
        \PY{k}{return}  \PY{l+s+s1}{\PYZsq{}}\PY{l+s+s1}{The function has a maximum.}\PY{l+s+s1}{\PYZsq{}}
        \PY{k}{return} \PY{l+s+s1}{\PYZsq{}}\PY{l+s+s1}{The function has no minimum or maximum.}\PY{l+s+s1}{\PYZsq{}}
    \end{Verbatim}
\end{tcolorbox}

\begin{tcolorbox}[breakable, size=fbox, boxrule=1pt, pad at break*=1mm,colback=cellbackground, colframe=cellborder]
    \prompt{In}{incolor}{10}{\hspace{4pt}}
    \begin{Verbatim}[commandchars=\\\{\}]
        \PY{n}{x} \PY{o}{=} \PY{n}{Symbol}\PY{p}{(}\PY{l+s+s1}{\PYZsq{}}\PY{l+s+s1}{x}\PY{l+s+s1}{\PYZsq{}}\PY{p}{)}
        \PY{n}{y} \PY{o}{=} \PY{n}{Symbol}\PY{p}{(}\PY{l+s+s1}{\PYZsq{}}\PY{l+s+s1}{y}\PY{l+s+s1}{\PYZsq{}}\PY{p}{)}
        \PY{n}{example\PYZus{}func\PYZus{}1} \PY{o}{=} \PY{n+nb}{eval}\PY{p}{(}\PY{l+s+s1}{\PYZsq{}}\PY{l+s+s1}{exp(x\PYZhy{}y)*(x**2\PYZhy{}2*y**2)}\PY{l+s+s1}{\PYZsq{}}\PY{p}{)}
        \PY{n}{example\PYZus{}func\PYZus{}2} \PY{o}{=} \PY{n+nb}{eval}\PY{p}{(}\PY{l+s+s1}{\PYZsq{}}\PY{l+s+s1}{x**2\PYZhy{}cos(y)}\PY{l+s+s1}{\PYZsq{}}\PY{p}{)}
        \PY{n+nb}{print}\PY{p}{(}\PY{l+s+s1}{\PYZsq{}}\PY{l+s+s1}{Gradient example using:}\PY{l+s+s1}{\PYZsq{}}\PY{p}{)}



        \PY{c+c1}{\PYZsh{} Case study maximum}
        \PY{n}{display}\PY{p}{(}\PY{n}{example\PYZus{}func\PYZus{}1}\PY{p}{)}
        \PY{n+nb}{print}\PY{p}{(}\PY{n}{is\PYZus{}critical\PYZus{}point}\PY{p}{(}\PY{n}{example\PYZus{}func\PYZus{}1}\PY{p}{,} \PY{p}{(}\PY{o}{\PYZhy{}}\PY{l+m+mi}{4}\PY{p}{,}\PY{o}{\PYZhy{}}\PY{l+m+mi}{2}\PY{p}{)}\PY{p}{)}\PY{p}{,} \PY{p}{(}\PY{o}{\PYZhy{}}\PY{l+m+mi}{4}\PY{p}{,} \PY{o}{\PYZhy{}}\PY{l+m+mi}{2}\PY{p}{)}\PY{p}{)}
        \PY{n+nb}{print}\PY{p}{(}\PY{n}{is\PYZus{}critical\PYZus{}point}\PY{p}{(}\PY{n}{example\PYZus{}func\PYZus{}1}\PY{p}{,} \PY{p}{(}\PY{l+m+mi}{0}\PY{p}{,} \PY{l+m+mi}{0}\PY{p}{)}\PY{p}{)}\PY{p}{,} \PY{p}{(}\PY{l+m+mi}{0}\PY{p}{,} \PY{l+m+mi}{0}\PY{p}{)}\PY{p}{)}
        \PY{n+nb}{print}\PY{p}{(}\PY{p}{)}
        \PY{n}{display}\PY{p}{(}\PY{n}{example\PYZus{}func\PYZus{}2}\PY{p}{)}
        \PY{n+nb}{print}\PY{p}{(}\PY{n}{is\PYZus{}critical\PYZus{}point}\PY{p}{(}\PY{n}{example\PYZus{}func\PYZus{}2}\PY{p}{,} \PY{p}{(}\PY{l+m+mi}{0}\PY{p}{,} \PY{n}{math}\PY{o}{.}\PY{n}{pi}\PY{p}{)}\PY{p}{)}\PY{p}{,} \PY{p}{(}\PY{l+m+mi}{0}\PY{p}{,} \PY{n}{math}\PY{o}{.}\PY{n}{pi}\PY{p}{)}\PY{p}{)}

        \PY{n+nb}{print}\PY{p}{(}\PY{n}{is\PYZus{}critical\PYZus{}point}\PY{p}{(}\PY{n}{example\PYZus{}func\PYZus{}2}\PY{p}{,} \PY{p}{(}\PY{l+m+mi}{0}\PY{p}{,} \PY{l+m+mi}{2}\PY{o}{*}\PY{n}{math}\PY{o}{.}\PY{n}{pi}\PY{p}{)}\PY{p}{)}\PY{p}{,} \PY{p}{(}\PY{l+m+mi}{0}\PY{p}{,} \PY{l+m+mi}{2}\PY{o}{*}\PY{n}{math}\PY{o}{.}\PY{n}{pi}\PY{p}{)}\PY{p}{)}
    \end{Verbatim}
\end{tcolorbox}

\begin{Verbatim}[commandchars=\\\{\}]
    Gradient example using:
\end{Verbatim}

$\displaystyle \left(x^{2} - 2 y^{2}\right) e^{x - y}$


\begin{Verbatim}[commandchars=\\\{\}]
    The function has no minimum or maximum. (-4, -2)
    The function has no minimum or maximum. (0, 0)

\end{Verbatim}

$\displaystyle x^{2} - \cos{\left(y \right)}$


\begin{Verbatim}[commandchars=\\\{\}]
    The function has a minimum. (0, 3.141592653589793)
    The function has a minimum. (0, 6.283185307179586)
\end{Verbatim}

\hypertarget{gradient-descent}{%
\subsection{Gradient Descent}\label{gradient-descent}}

We implement the Gradient Descent to find if there is any critical
points and if there is we applied previous functions define before.\\
Then we determine if the critical point is either a minimum or a maximum
based on the parameter \textbf{minimize} of our function.

If it's true, then we are looking for a minimum point and if it is
false, we are looking for a maximum

\begin{tcolorbox}[breakable, size=fbox, boxrule=1pt, pad at break*=1mm,colback=cellbackground, colframe=cellborder]
    \prompt{In}{incolor}{11}{\hspace{4pt}}
    \begin{Verbatim}[commandchars=\\\{\}]
        \PY{k}{def} \PY{n+nf}{gradient\PYZus{}descent}\PY{p}{(}\PY{n}{func}\PY{p}{,} \PY{n}{starting\PYZus{}point}\PY{p}{,} \PY{n}{alpha}\PY{o}{=}\PY{l+m+mf}{0.02}\PY{p}{,} \PY{n}{minimize}\PY{o}{=}\PY{k+kc}{True}\PY{p}{)}\PY{p}{:}
        \PY{n+nb}{print}\PY{p}{(}\PY{l+s+s1}{\PYZsq{}}\PY{l+s+s1}{Starting gradient descent for:}\PY{l+s+s1}{\PYZsq{}}\PY{p}{)}
        \PY{n}{display}\PY{p}{(}\PY{n}{func}\PY{p}{)}
        \PY{n}{iterations} \PY{o}{=} \PY{l+m+mi}{3000}
        \PY{n}{epsilon} \PY{o}{=} \PY{l+m+mf}{0.001}
        \PY{n+nb}{print}\PY{p}{(}\PY{n}{f}\PY{l+s+s1}{\PYZsq{}}\PY{l+s+s1}{Settings: alpha=}\PY{l+s+si}{\PYZob{}alpha\PYZcb{}}\PY{l+s+s1}{, epsilon=}\PY{l+s+si}{\PYZob{}epsilon\PYZcb{}}\PY{l+s+s1}{, iterations=}\PY{l+s+si}{\PYZob{}iterations\PYZcb{}}\PY{l+s+s1}{\PYZsq{}}\PY{p}{)}
        \PY{n}{mode} \PY{o}{=} \PY{l+s+s1}{\PYZsq{}}\PY{l+s+s1}{minimize}\PY{l+s+s1}{\PYZsq{}} \PY{k}{if} \PY{p}{(}\PY{n}{minimize}\PY{p}{)} \PY{k}{else} \PY{l+s+s1}{\PYZsq{}}\PY{l+s+s1}{maximize}\PY{l+s+s1}{\PYZsq{}}
        \PY{n+nb}{print}\PY{p}{(}\PY{n}{f}\PY{l+s+s1}{\PYZsq{}}\PY{l+s+s1}{Mode: }\PY{l+s+si}{\PYZob{}mode\PYZcb{}}\PY{l+s+s1}{\PYZsq{}}\PY{p}{)}
        \PY{n}{optim\PYZus{}func} \PY{o}{=} \PY{n}{func} \PY{k}{if} \PY{n}{minimize} \PY{k}{else} \PY{n}{func}\PY{o}{*}\PY{p}{(}\PY{o}{\PYZhy{}}\PY{l+m+mi}{1}\PY{p}{)}

        \PY{n}{point} \PY{o}{=} \PY{n}{starting\PYZus{}point}
        \PY{n}{syms} \PY{o}{=} \PY{n+nb}{list}\PY{p}{(}\PY{n}{func}\PY{o}{.}\PY{n}{free\PYZus{}symbols}\PY{p}{)}

        \PY{n}{gradient\PYZus{}function} \PY{o}{=} \PY{n}{Matrix}\PY{p}{(}\PY{n}{gradient}\PY{p}{(}\PY{n}{optim\PYZus{}func}\PY{p}{,} \PY{n}{syms}\PY{p}{)}\PY{p}{)}
        \PY{n}{display}\PY{p}{(}\PY{n}{gradient\PYZus{}function}\PY{p}{)}
        \PY{k}{for} \PY{n}{i} \PY{o+ow}{in} \PY{n+nb}{range}\PY{p}{(}\PY{n}{iterations}\PY{p}{)}\PY{p}{:}
        \PY{n}{grad} \PY{o}{=} \PY{n}{evaluate\PYZus{}value}\PY{p}{(}\PY{n}{gradient\PYZus{}function}\PY{p}{,} \PY{n}{syms}\PY{p}{,} \PY{n}{point}\PY{p}{)}
        \PY{n}{point} \PY{o}{=} \PY{p}{(}\PY{n}{point}\PY{o}{\PYZhy{}}\PY{n}{alpha}\PY{o}{*}\PY{n}{grad}\PY{p}{)}\PY{o}{.}\PY{n}{evalf}\PY{p}{(}\PY{p}{)}
        \PY{k}{if} \PY{n}{grad}\PY{o}{.}\PY{n}{norm}\PY{p}{(}\PY{p}{)} \PY{o}{\PYZlt{}} \PY{n}{epsilon}\PY{p}{:}
        \PY{n+nb}{print}\PY{p}{(}\PY{n}{f}\PY{l+s+s1}{\PYZsq{}}\PY{l+s+s1}{Gradient small enough, terminating at iteration }\PY{l+s+si}{\PYZob{}i\PYZcb{}}\PY{l+s+s1}{\PYZsq{}}\PY{p}{)}
        \PY{k}{break}
        \PY{n+nb}{print}\PY{p}{(}\PY{l+s+s1}{\PYZsq{}}\PY{l+s+s1}{Descent finished, at point:}\PY{l+s+s1}{\PYZsq{}}\PY{p}{,} \PY{n}{point}\PY{p}{)}
        \PY{n+nb}{print}\PY{p}{(}\PY{n}{is\PYZus{}critical\PYZus{}point}\PY{p}{(}\PY{n}{func}\PY{p}{,} \PY{n}{point}\PY{p}{)}\PY{p}{)}
    \end{Verbatim}
\end{tcolorbox}

For the example 1 :

\begin{tcolorbox}[breakable, size=fbox, boxrule=1pt, pad at break*=1mm,colback=cellbackground, colframe=cellborder]
    \prompt{In}{incolor}{12}{\hspace{4pt}}
    \begin{Verbatim}[commandchars=\\\{\}]
        \PY{n}{start\PYZus{}point} \PY{o}{=} \PY{n}{Matrix}\PY{p}{(}\PY{p}{[}\PY{o}{\PYZhy{}}\PY{l+m+mi}{1}\PY{p}{,} \PY{o}{\PYZhy{}}\PY{l+m+mi}{1}\PY{p}{]}\PY{p}{)}
        \PY{n}{gradient\PYZus{}descent}\PY{p}{(}\PY{n}{example\PYZus{}func\PYZus{}1}\PY{p}{,} \PY{n}{start\PYZus{}point}\PY{p}{,} \PY{n}{minimize}\PY{o}{=}\PY{k+kc}{False}\PY{p}{)}
    \end{Verbatim}
\end{tcolorbox}

\begin{Verbatim}[commandchars=\\\{\}]
    Starting gradient descent for:
\end{Verbatim}

$\displaystyle \left(x^{2} - 2 y^{2}\right) e^{x - y}$


\begin{Verbatim}[commandchars=\\\{\}]
    Settings: alpha=0.02, epsilon=0.001, iterations=3000
    Mode: maximize
\end{Verbatim}

$\displaystyle \left[\begin{matrix}4 y e^{x - y} - \left(- x^{2} + 2 y^{2}\right) e^{x - y}\\- 2 x e^{x - y} + \left(- x^{2} + 2 y^{2}\right) e^{x - y}\end{matrix}\right]$


\begin{Verbatim}[commandchars=\\\{\}]
    Descent finished, at point: Matrix([[-1.97623271798963], [-3.96569690543399]])
    The function has a maximum.
\end{Verbatim}

For the example 2 :

\begin{tcolorbox}[breakable, size=fbox, boxrule=1pt, pad at break*=1mm,colback=cellbackground, colframe=cellborder]
    \prompt{In}{incolor}{13}{\hspace{4pt}}
    \begin{Verbatim}[commandchars=\\\{\}]
        \PY{n}{start\PYZus{}point} \PY{o}{=} \PY{n}{Matrix}\PY{p}{(}\PY{p}{[}\PY{l+m+mi}{1}\PY{p}{,} \PY{l+m+mi}{2}\PY{p}{]}\PY{p}{)}
        \PY{n}{gradient\PYZus{}descent}\PY{p}{(}\PY{n}{example\PYZus{}func\PYZus{}2}\PY{p}{,} \PY{n}{start\PYZus{}point}\PY{p}{,} \PY{n}{minimize}\PY{o}{=}\PY{k+kc}{True}\PY{p}{)}
    \end{Verbatim}
\end{tcolorbox}

\begin{Verbatim}[commandchars=\\\{\}]
    Starting gradient descent for:
\end{Verbatim}

$\displaystyle x^{2} - \cos{\left(y \right)}$


\begin{Verbatim}[commandchars=\\\{\}]
    Settings: alpha=0.02, epsilon=0.001, iterations=3000
    Mode: minimize
\end{Verbatim}

$\displaystyle \left[\begin{matrix}\sin{\left(y \right)}\\2 x\end{matrix}\right]$


\begin{Verbatim}[commandchars=\\\{\}]
    Gradient small enough, terminating at iteration 347
    Descent finished, at point: Matrix([[0.000968939941592121],
    [1.35337794065227e-6]])
    The function has a minimum.
\end{Verbatim}


    % Add a bibliography block to the postdoc



\end{document}
