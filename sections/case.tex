\documentclass[../convex_optimization.tex]{subfiles}
\newcommand{\Z}{\mathbb{Z}}
\begin{document}
We will consider two cases, two different functions $f$.
\subsection{First study}
Consider the function as given by (\ref{firstfunc}):
\begin{equation}
    f(x, y)=e^{x-y} (x^2 - 2y^2)
    \label{firstfunc}
\end{equation}
First  we will calculate the gradient of $f$:
\begin{equation}
    \nabla f(x,y) = 
    \begin{bmatrix}
    e^{x-y} (x^2 - 2y^2 + 2x) \\
    e^{x-y} (-x^2 + 2y^2 - 4y)
    \end{bmatrix}
    \label{firstgrad}
\end{equation}
Since the exponential is strictly positive, we can determine the critical points by solving the following system:
\begin{equation}
    \systeme{0 = x^2 - 2y^2 + 2x,0 = -x^2 + 2y^2 - 4y}
    \label{firstsystem}
\end{equation}


This system is easy to solve and we find  the following critical points :
$p_0 = (0, 0)$, $p_1 = (-4, -2)$.
Now we will calculate the partial derivatives and the Hessian matrix
in order to evaluate it on the critical points
\begin{align}
\frac{\partial f}{\partial x^2} &= e^{x-y} (x^2- 2y^2 + 4x + 2)\nonumber\\
\frac{\partial f}{\partial y^2} &=  e^{x-y} (x^2- 2y^2 + 8y - 4)\nonumber\\
\frac{\partial f}{\partial x \partial y } &= e^{x-y} (-x^2 +2y^2 - 2x - 4y)\nonumber
\end{align}
So by (\ref{hessian_definition}) and using that in this case
$\frac{\partial f}{\partial x \partial y }=\frac{\partial f}{\partial y \partial x }$, 
we have:
\begin{equation}
    \mathbf H_f(x,y) =\begin{pmatrix}
        e^{x-y} (x^2- 2y^2 + 4x + 2) &   e^{x-y} (-x^2 +2y^2 - 2x - 4y) \\
        e^{x-y} (-x^2 +2y^2 - 2x - 4y) & e^{x-y} (x^2- 2y^2 + 8y - 4)
    \end{pmatrix}
    \label{firsthessian}
\end{equation}
We can now determine for each critical point what type it is; maximum, minimum, or saddle.
\begin{enumerate}
\item $p_0$
    \begin{gather}
        \mathbf H_f(0,0) =
        \begin{pmatrix}
            2  & 0\\
            0 & -4
        \end{pmatrix}\nonumber\\
        \begin{vmatrix}
            2  & 0\\
            0 & -4
        \end{vmatrix} = - 8 < 0
        \nonumber
    \end{gather}

Since the determinant of the hessian matrix is negative,
its eigenvalues are not zero.
One being positive and the other one being negative means that the matrix
is neither positive definite nor negative definite
(by (\ref{positive_definite})).
We can then conclude that $p_0$ is a saddle point.
\item $p_1$
    \begin{gather}
        \mathbf H_f(-4,-2) =
        \begin{pmatrix}
            -6e^{ -2}  & 8e^{ -2}\\
            8e^{ -2} & -12e^{ -2}
        \end{pmatrix}\nonumber\\
        \begin{vmatrix}
            -6e^{ -2}  & 8e^{ -2}\\
            8e^{ -2} & -12e^{ -2}
        \end{vmatrix} =  8e^{ -4} > 0
        \nonumber
    \end{gather}

Then we calculate its trace :
\end{enumerate}
\begin{equation}
    Tr(\mathbf H_f(-4,-2)) = -18e^ {-2 }<0 
    \nonumber
\end{equation}
Since the determinant is positive and the trace is negative we know that
both the eigenvalues are negative which means that
$p_1$ is a maximum of $f$.


\subsection{Second case}
For this next case, we will use a different $f$:
\begin{equation}
    f(x, y )=x^2 - \cos(y)
\end{equation}
First  we will calculate the gradient of $f$:
\begin{equation}
\nabla f(x,y) = 
\begin{bmatrix}
    2x \\ \sin(x)
\end{bmatrix}
\end{equation}
We can determine the critical points by solving the following system:
\begin{equation}
    \systeme{0 = 2x,0 = \sin(x)}
    \label{sinsys}
\end{equation}
By (\ref{sinsys}) we get the critical points
$p_k = (0, k\pi), \forall k \in \Z$

We can calculate the partial derivatives and the Hessian matrix similar to before:
\begin{align}
   \frac{\partial f}{\partial x^2} &= 2\nonumber & \quad
   \frac{\partial f}{\partial y^2} &= \cos(y)\nonumber & \quad
\frac{\partial f}{\partial x \partial y } &= 0 \nonumber 
\end{align}
And by 
$\frac{\partial f}{\partial x \partial y }=\frac{\partial f}{\partial y \partial x }$, we get
\begin{equation}
   \mathbf H_f(x,y) =\begin{pmatrix}
       2  & 0\\
       0 & \cos(y)
   \end{pmatrix}
\end{equation}

We can now study 2 cases for the evaluation of the Hessian matrix, $k$ even and $k$ odd.
\begin{equation}
    \mathbf H_f(0, k\pi) =\begin{pmatrix}
        2  & 0\\
        0 & \cos(k\pi)
    \end{pmatrix}\nonumber
\end{equation}

\begin{enumerate}
    \item $k\text{ even} \iff k=2n, n\in \mathbb Z \Rightarrow \cos(k\pi) = 1$
        \begin{equation}
            \det{(\mathbf H_f(0, 2n\pi))}=
            \begin{vmatrix}
                2  & 0\\
                0 &1
            \end{vmatrix} = 2>0\nonumber
        \end{equation}

We then have the eigenvalues of the same sign. 
\begin{equation}
    Tr(\mathbf H_f(0,k\pi)) = 3>0 \nonumber
\end{equation}
Then \(p_k\) is a minimum for $f$.

    \item $k\text{ odd} \iff k=2n+1, n\in \mathbb Z \Rightarrow \cos(k\pi) = -1$
\end{enumerate}
\begin{equation}
    \det{(\mathbf H_f(0, (2n+1)\pi))}=
    \begin{vmatrix}
        2  & 0\\
        0 & -1
    \end{vmatrix} = -2<0\nonumber
\end{equation}
Since the determinant of the Hessian is negative, its eigenvalues are not zero.
One being positive and the other one being negative means that the matrix is
neither definite positive nor definite negative. We conclude that $p_k$ is a saddle point.

To conclude :
\begin{itemize}
    \item If $k$ is even then $p_k$ is a minimum for $f$
    \item If $k$ is odd then $p_k$ is a saddle point for $f$
\end{itemize}
\end{document}
